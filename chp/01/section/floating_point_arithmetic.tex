\section{Floating point}\label{sec:floating-point}

A floating point number consists of a sign bit, a significand, also called the
mantissa, and a power of a fixed base. The general form is this:
\begin{equation}\label{eq:floating-point-representation}
  \pm{.d_{1}\ldots{d_{m}}\times{B^{e}}},
\end{equation}
where
\begin{table}[H]
  \centering
  % \rowcolors{1}{white}{gray!15}
  \begin{tabular}{ll}
    \textbf{number of digits} & \(m\in\integers\), \(m>0\),           \\
    \textbf{base}             & \(B\in\integers\), \(B\geqslant{2}\), \\
    \textbf{exponent}         & \(e\in\integers\),                    \\
    \textbf{digits}           & \(d_{1},\ldots,d_{m}\in\left\{z\in\integers:0\leqslant{z}\leqslant{B-1}\right\}\), \(d_{1}\neq{0}\).
  \end{tabular}
\end{table}
For example, let's imagine a hypothetical computer in which
\begin{equation}\label{eq:hypothetical-computer}
  m=2,\quad{B=2}\quad\text{and}\quad{-1\leqslant{e}\leqslant{2}}.
\end{equation}
Then, the only positive numbers that are representable on such a machine are
listed below:
\begin{table}[H]
  \centering
  \rowcolors{1}{white}{gray!15}
  \begin{tabular}{l|l||l|l}
    \textbf{binary}       & \textbf{decimal} & \textbf{binary}      & \textbf{decimal} \\ \hline
    \(.10\times{2^{-1}}\) & 1/4              & \(.10\times{2^{0}}\) & 1/2              \\
    \(.11\times{2^{-1}}\) & 3/8              & \(.11\times{2^{0}}\) & 3/4              \\
    \(.10\times{2^{1}}\)  & 1                & \(.10\times{2^{2}}\) & 2                \\
    \(.11\times{2^{1}}\)  & 3/2              & \(.11\times{2^{2}}\) & 3                \\
  \end{tabular}
\end{table}

\acrshort{gnu} C uses the floating point representations specified by the
\acrshort{ieee} 754-2008 Standard for Floating Point Arithmetic~\cite{4610935}.

\section{Floating point arithmetic}\label{sec:floating-point-arithmetic}

In this section we explore the precision of floating point arithmetic
operations. For pedagogical reasons, we are going to work with another
hypothetical computer, one in which \(m=2\), \(B=2\) and
\(-5\leqslant{e}\leqslant{5}\).

\begin{example}
  Sum \(4.32\) and \(0.064\).
\end{example}


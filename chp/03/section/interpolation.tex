\section{Interpolation}

Sejam \((x_{i},y_{i})\in\euclideanspace{2}\), \(i\in\left\{0,1,2\right\}\),
pontos dois a dois distintos. Podemos supor, possivelmente após renomear os
\((x_{i},y_{i})\), que \(x_{0}<x_{1}<x_{2}\). Queremos encontrar um polinômio
do segundo grau e com coeficientes em \(\reals\), digamos,
\(p{\left(x\right)}=a_{0}+a_{1}x+a_{2}x^{2}\), tal que \(p(x_{i})=y_{i}\), para
todo \(i\in\left\{0,1,2\right\}\). Para tanto, basta que se resolva ao sistema
linear
\begin{equation}\label{eq:sistema-linear-coeficientes-polinomio-interpolador-de-grau-dois}
  \begin{bmatrix}
    1 & x_{0} & x_{0}^{2} \\
    1 & x_{1} & x_{1}^{2} \\
    1 & x_{2} & x_{2}^{2}
  \end{bmatrix}
  \begin{bmatrix}
    a_{0} \\ a_{1} \\ a_{2}
  \end{bmatrix}
  =
  \begin{bmatrix}
    y_{0} \\ y_{1} \\ y_{2}
  \end{bmatrix}
\end{equation}
Caso os pontos \(x_{0},x_{1}\) e \(x_{2}\) se encontrem em progressão
aritmética, isto é, se existe \(h>0\) tal que \(x_{i+1}-x_{i}=h\) para todo
\(i\in\left\{0,1,2\right\}\), então nós podemos dar uma expressão fechada para
a solução
de~\eqref{eq:sistema-linear-coeficientes-polinomio-interpolador-de-grau-dois}
em termos \(x_{0},h,y_{0},y_{1}\) e \(y_{2}\), apenas. É isto o que passamos
agora a descrever.

Seja \(X_{ij}\in\euclideanspace{2\times{2}}\) a matriz que se obtém de
\begin{equation*}
  X=
  \begin{bmatrix}
    1 & x_{0} & x_{0}^{2} \\
    1 & x_{1} & x_{1}^{2} \\
    1 & x_{2} & x_{2}^{2}
  \end{bmatrix}
\end{equation*}
através da deleção de suas linha \(i\) e coluna \(j\), para todos
\(i,j\in\left\{0,1,2\right\}\). Como se tem
\begin{equation*}
  \det{\left(X\right)}
  =
  x_{1}x_{2}^{2}+x_{0}x_{1}^{2}+x_{0}^{2}x_{2}
  -x_{1}x_{0}^{2}-x_{2}x_{1}^{2}-x_{2}^{2}x_{0}
  =2h^{3}>0,
\end{equation*}
sabe-se que existe a matriz inversa
\begin{equation*}
  X^{-1}=\frac{1}{\det{\left(X\right)}}\left[(-1)^{i+j}\det{\left(X_{ij}\right)}\right]_{0\leqslant{i,j}\leqslant{2}}
\end{equation*}
que, por um cálculo direto, se mostra ser igual a
\begin{equation}\label{eq:matriz-inversa-da-matriz-de-coeficientes-do-sistema}
  X^{-1}
  =
  \frac{1}{2h^{3}}
  \begin{bmatrix}
    x_{0}^{2}h+3x_{0}h^{2}+2h^{3} & -2x_{0}^{2}h-4x_{0}h^{2} & x_{0}^{2}h+x_{0}h^{2} \\
    -2x_{0}y-2h^{2}               & 4x_{0}h+4h^{2}           & -2x_{0}h-h^{2}        \\
    h                             & -2h                      & h
  \end{bmatrix}
\end{equation}
Finalmente, tem-se que
\begin{equation}\label{eq:coeficientes-do-polinomio-interpolador-de-grau-dois}
  \begin{bmatrix}
    a_{0} \\ a_{1} \\ a_{2}
  \end{bmatrix}
  =
  X^{-1}
  \begin{bmatrix}
    y_{0} \\ y_{1} \\ y_{2}
  \end{bmatrix}
\end{equation}
por~\eqref{eq:sistema-linear-coeficientes-polinomio-interpolador-de-grau-dois}
e~\eqref{eq:matriz-inversa-da-matriz-de-coeficientes-do-sistema}.
\begin{example}
  Sejam \((x_{0},y_{0})\), \(i\in\left\{0,1,2\right\}\), dados de acordo com a
  tabela abaixo:
  \[
    \rowcolors{1}{white}{gray!15}
    \begin{array}{r|r|r}
      i & x_{i} & y_{i} \\ \hline
      0 &   1.0 &   0.2 \\
      1 &   2.0 &   2.0 \\
      2 &   3.0 &  -1.5
    \end{array}
  \]
  Deste modo,
  por~\eqref{eq:coeficientes-do-polinomio-interpolador-de-grau-dois}, tem-se
  que
  \begin{equation}\label{eq:coeficientes-polinomio-interpolador-de-grau-dois-para-o-primeiro-exemplo}
    \begin{bmatrix}
      a_{0} \\ a_{1} \\ a_{2}
    \end{bmatrix}
    =
    \begin{bmatrix}
      3.0 & -3.0 &  0.5 \\
      -2.0 &  4.0 & -1.5 \\
      0.5 & -2.0 &  0.5
    \end{bmatrix}
    \begin{bmatrix}
      0.2 \\ 2.0 \\ -1.5
    \end{bmatrix}
  \end{equation}
  e, assim, que
  \(p{\left(x\right)}=\frac{138}{20}+\frac{197}{20}x-\frac{53}{20}x^{2}\).
\end{example}
